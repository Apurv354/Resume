
%-----------------------------------------------------------------------------------------------------------------------------------------------%
%	The MIT License (MIT)
%
%	Copyright (c) 2021 Philip Empl
%
%	Permission is hereby granted, free of charge, to any person obtaining a copy
%	of this software and associated documentation files (the "Software"), to deal
%	in the Software without restriction, including without limitation the rights
%	to use, copy, modify, merge, publish, distribute, sublicense, and/or sell
%	copies of the Software, and to permit persons to whom the Software is
%	furnished to do so, subject to the following conditions:
%	
%	THE SOFTWARE IS PROVIDED "AS IS", WITHOUT WARRANTY OF ANY KIND, EXPRESS OR
%	IMPLIED, INCLUDING BUT NOT LIMITED TO THE WARRANTIES OF MERCHANTABILITY,
%	FITNESS FOR A PARTICULAR PURPOSE AND NONINFRINGEMENT. IN NO EVENT SHALL THE
%	AUTHORS OR COPYRIGHT HOLDERS BE LIABLE FOR ANY CLAIM, DAMAGES OR OTHER
%	LIABILITY, WHETHER IN AN ACTION OF CONTRACT, TORT OR OTHERWISE, ARISING FROM,
%	OUT OF OR IN CONNECTION WITH THE SOFTWARE OR THE USE OR OTHER DEALINGS IN
%	THE SOFTWARE.
%	
%
%-----------------------------------------------------------------------------------------------------------------------------------------------%


%============================================================================%
%
%	DOCUMENT DEFINITION
%
%============================================================================%

\documentclass[10pt,A4,english]{article}	


%----------------------------------------------------------------------------------------
%	ENCODING
%----------------------------------------------------------------------------------------

% we use utf8 since we want to build from any machine
\usepackage[utf8]{inputenc}		
\usepackage[USenglish]{isodate}
\usepackage{fancyhdr}
\usepackage[numbers]{natbib}

%----------------------------------------------------------------------------------------
%	LOGIC
%----------------------------------------------------------------------------------------

% provides \isempty test
\usepackage{xstring, xifthen}
\usepackage{enumitem}
\usepackage[english]{babel}
\usepackage{blindtext}
\usepackage{pdfpages}
\usepackage{changepage}
%----------------------------------------------------------------------------------------
%	FONT BASICS
%----------------------------------------------------------------------------------------

% some tex-live fonts - choose your own

%\usepackage[defaultsans]{droidsans}
%\usepackage[default]{comfortaa}
%\usepackage{cmbright}
\usepackage[default]{raleway}
%\usepackage{fetamont}
%\usepackage[default]{gillius}
%\usepackage[light,math]{iwona}
%\usepackage[thin]{roboto} 

% set font default
\renewcommand*\familydefault{\sfdefault} 	
\usepackage[T1]{fontenc}

% more font size definitions
\usepackage{moresize}

%----------------------------------------------------------------------------------------
%	FONT AWESOME ICONS
%---------------------------------------------------------------------------------------- 

% include the fontawesome icon set
\usepackage{fontawesome}

% use to vertically center content
% credits to: http://tex.stackexchange.com/questions/7219/how-to-vertically-center-two-images-next-to-each-other
\newcommand{\vcenteredinclude}[1]{\begingroup
\setbox0=\hbox{\includegraphics{#1}}%
\parbox{\wd0}{\box0}\endgroup}
\newcommand{\tab}[1]{\hspace{.2\textwidth}\rlap{#1}}
% use to vertically center content
% credits to: http://tex.stackexchange.com/questions/7219/how-to-vertically-center-two-images-next-to-each-other
\newcommand*{\vcenteredhbox}[1]{\begingroup
\setbox0=\hbox{#1}\parbox{\wd0}{\box0}\endgroup}

% icon shortcut
\newcommand{\icon}[3] { 							
	\makebox(#2, #2){\textcolor{maincol}{\csname fa#1\endcsname}}
}	


% icon with text shortcut
\newcommand{\icontext}[4]{ 						
	\vcenteredhbox{\icon{#1}{#2}{#3}}  \hspace{2pt}  \parbox{0.9\mpwidth}{\textcolor{#4}{#3}}
}

% icon with website url
\newcommand{\iconhref}[5]{ 						
    \vcenteredhbox{\icon{#1}{#2}{#5}}  \hspace{2pt} \href{#4}{\textcolor{#5}{#3}}
}

% icon with email link
\newcommand{\iconemail}[5]{ 						
    \vcenteredhbox{\icon{#1}{#2}{#5}}  \hspace{2pt} \href{mailto:#4}{\textcolor{#5}{#3}}
}

%----------------------------------------------------------------------------------------
%	PAGE LAYOUT  DEFINITIONS
%----------------------------------------------------------------------------------------

% page outer frames (debug-only)
% \usepackage{showframe}		

% we use paracol to display breakable two columns
\usepackage{paracol}
\usepackage{tikzpagenodes}
\usetikzlibrary{calc}
\usepackage{lmodern}
\usepackage{multicol}
\usepackage{lipsum}
\usepackage{atbegshi}
% define page styles using geometry
\usepackage[a4paper]{geometry}

% remove all possible margins
\geometry{top=1cm, bottom=1cm, left=1cm, right=1cm}

\usepackage{fancyhdr}
\pagestyle{empty}

% space between header and content
% \setlength{\headheight}{0pt}

% indentation is zero
\setlength{\parindent}{0mm}

%----------------------------------------------------------------------------------------
%	TABLE /ARRAY DEFINITIONS
%---------------------------------------------------------------------------------------- 

% extended aligning of tabular cells
\usepackage{array}

% custom column right-align with fixed width
% use like p{size} but via x{size}
\newcolumntype{x}[1]{%
>{\raggedleft\hspace{0pt}}p{#1}}%


%----------------------------------------------------------------------------------------
%	GRAPHICS DEFINITIONS
%---------------------------------------------------------------------------------------- 

%for header image
\usepackage{graphicx}

% use this for floating figures
% \usepackage{wrapfig}
% \usepackage{float}
% \floatstyle{boxed} 
% \restylefloat{figure}

%for drawing graphics		
\usepackage{tikz}			
\usepackage{ragged2e}	
\usetikzlibrary{shapes, backgrounds,mindmap, trees}

%----------------------------------------------------------------------------------------
%	Color DEFINITIONS
%---------------------------------------------------------------------------------------- 
\usepackage{transparent}
\usepackage{color}

% primary color
\definecolor{maincol}{RGB}{ 64,64,64}

% accent color, secondary
% \definecolor{accentcol}{RGB}{ 250, 150, 10 }

% dark color
\definecolor{darkcol}{RGB}{ 70, 70, 70 }

% light color
\definecolor{lightcol}{RGB}{245,245,245}

\definecolor{accentcol}{RGB}{59,77,97}



% Package for links, must be the last package used
\usepackage[hidelinks]{hyperref}

% returns minipage width minus two times \fboxsep
% to keep padding included in width calculations
% can also be used for other boxes / environments
\newcommand{\mpwidth}{\linewidth-\fboxsep-\fboxsep}
	


%============================================================================%
%
%	CV COMMANDS
%
%============================================================================%

%----------------------------------------------------------------------------------------
%	 CV LIST
%----------------------------------------------------------------------------------------

% renders a standard latex list but abstracts away the environment definition (begin/end)
\newcommand{\cvlist}[1] {
	\begin{itemize}{#1}\end{itemize}
}

%----------------------------------------------------------------------------------------
%	 CV TEXT
%----------------------------------------------------------------------------------------

% base class to wrap any text based stuff here. Renders like a paragraph.
% Allows complex commands to be passed, too.
% param 1: *any
\newcommand{\cvtext}[1] {
	\begin{tabular*}{1\mpwidth}{p{0.98\mpwidth}}
		\parbox{1\mpwidth}{#1}
	\end{tabular*}
}
\newcommand{\cvtextsmall}[1] {
	\begin{tabular*}{0.8\mpwidth}{p{0.8\mpwidth}}
		\parbox{0.8\mpwidth}{#1}
	\end{tabular*}
}
%----------------------------------------------------------------------------------------
%	CV SECTION
%----------------------------------------------------------------------------------------

% Renders a a CV section headline with a nice underline in main color.
% param 1: section title
\newcommand{\cvsection}[1] {
	\vspace{14pt}
	\cvtext{
		\textbf{\LARGE{\textcolor{darkcol}{#1}}}\\[-4pt]
		\textcolor{accentcol}{ \rule{0.2\textwidth}{1.5pt} } \\
	}
}

\newcommand{\cvsectionsmall}[1] {
	\vspace{14pt}
	\cvtext{
		\textbf{\Large{\textcolor{darkcol}{#1}}}\\[-4pt]
		\textcolor{accentcol}{ \rule{0.2\textwidth}{1.5pt} } \\
	}
}

\newcommand{\cvheadline}[1] {
	\vspace{16pt}
	\cvtext{
		\textbf{\Huge{\textcolor{accentcol}{#1}}}\\[-4pt]
		 
	}
}

\newcommand{\cvsubheadline}[1] {
	\vspace{16pt}
	\cvtext{
		\textbf{\huge{\textcolor{darkcol}{#1}}}\\[-4pt]
		 
	}
}
%----------------------------------------------------------------------------------------
%	META SKILL
%----------------------------------------------------------------------------------------

% Renders a progress-bar to indicate a certain skill in percent.
% param 1: name of the skill / tech / etc.
% param 2: level (for example in years)
% param 3: percent, values range from 0 to 1
\newcommand{\cvskill}[3] {
	\begin{tabular*}{1\mpwidth}{p{0.72\mpwidth}  r}
 		\textcolor{black}{\textbf{#1}} & \textcolor{maincol}{#2}\\
	\end{tabular*}%
	
	\hspace{4pt}
	\begin{tikzpicture}[scale=1,rounded corners=2pt,very thin]
		\fill [lightcol] (0,0) rectangle (1\mpwidth, 0.15);
		\fill [accentcol] (0,0) rectangle (#3\mpwidth, 0.15);
  	\end{tikzpicture}%
}


%----------------------------------------------------------------------------------------
%	 CV EVENT
%----------------------------------------------------------------------------------------

% Renders a table and a paragraph (cvtext) wrapped in a parbox (to ensure minimum content
% is glued together when a pagebreak appears).
% Additional Information can be passed in text or list form (or other environments).
% the work you did
% param 1: time-frame i.e. Sep 14 - Jan 15 etc.
% param 2:	 event name (job position etc.)
% param 3: Customer, Employer, Industry
% param 4: Short description
% param 5: work done (optional)
% param 6: technologies include (optional)
% param 7: achievements (optional)
\newcommand{\cvevent}[7] {
	
	% we wrap this part in a parbox, so title and description are not separated on a pagebreak
	% if you need more control on page breaks, remove the parbox
	\parbox{\mpwidth}{
		\begin{tabular*}{1\mpwidth}{p{0.66\mpwidth}  r}
	 		\textcolor{black}{\textbf{#2}} & \colorbox{accentcol}{\makebox[0.32\mpwidth]{\textcolor{white}{\textbf{#1}}}} \\
			\textcolor{maincol}{#3} & \\
		\end{tabular*}\\[8pt]
	
		\ifthenelse{\isempty{#4}}{}{
			\cvtext{#4}\\
		}
	}
	\vspace{14pt}
}


%----------------------------------------------------------------------------------------
%	 CV META EVENT
%----------------------------------------------------------------------------------------

% Renders a CV event on the sidebar
% param 1: title
% param 2: subtitle (optional)
% param 3: customer, employer, etc,. (optional)
% param 4: info text (optional)
\newcommand{\cvmetaevent}[4] {
	\textcolor{maincol} { \cvtext{\textbf{\begin{flushleft}#1\end{flushleft}}}}

	\ifthenelse{\isempty{#2}}{}{
	\textcolor{black} {\cvtext{\textbf{#2}} }
	}

	\ifthenelse{\isempty{#3}}{}{
		\cvtext{{ \textcolor{maincol} {#3} }}\\
	}

	\cvtext{#4}\\[14pt]
}

%---------------------------------------------------------------------------------------
%	QR CODE
%----------------------------------------------------------------------------------------

% Renders a qrcode image (centered, relative to the parentwidth)
% param 1: percent width, from 0 to 1
\newcommand{\cvqrcode}[1] {
	\begin{center}
		\includegraphics[width={#1}\mpwidth]{qrcode}
	\end{center}
}


% HEADER AND FOOOTER 
%====================================
\newcommand\Header[1]{%
\begin{tikzpicture}[remember picture,overlay]
\fill[accentcol]
  (current page.north west) -- (current page.north east) --
  ([yshift=50pt]current page.north east|-current page text area.north east) --
  ([yshift=50pt,xshift=-3cm]current page.north|-current page text area.north) --
  ([yshift=10pt,xshift=-5cm]current page.north|-current page text area.north) --
  ([yshift=10pt]current page.north west|-current page text area.north west) -- cycle;
\node[font=\sffamily\bfseries\color{white},anchor=west,
  xshift=0.7cm,yshift=-0.32cm] at (current page.north west)
  {\fontsize{12}{12}\selectfont {#1}};
\end{tikzpicture}%
}

\newcommand\Footer[1]{%
\begin{tikzpicture}[remember picture,overlay]
\fill[lightcol]
  (current page.south east) -- (current page.south west) --
  ([yshift=-80pt]current page.south east|-current page text area.south east) --
  ([yshift=-80pt,xshift=-6cm]current page.south|-current page text area.south) --
  ([xshift=-2.5cm,yshift=-10pt]current page.south|-current page text area.south) --	
  ([yshift=-10pt]current page.south east|-current page text area.south east) -- cycle;
\node[yshift=0.32cm,xshift=9cm] at (current page.south) {\fontsize{10}{10}\selectfont \textbf{\thepage}};
\end{tikzpicture}%
}


%=====================================
%============================================================================%
%
%
%
%	DOCUMENT CONTENT
%
%
%
%============================================================================%
\begin{document}

\columnratio{0.31}
\setlength{\columnsep}{2.2em}
\setlength{\columnseprule}{4pt}
\colseprulecolor{white}


% LEBENSLAUF HIERE
\AtBeginShipoutFirst{\Header{CV}\Footer{1}}
\AtBeginShipout{\AtBeginShipoutAddToBox{\Header{CV}\Footer{2}}}

\newpage

\colseprulecolor{lightcol}
\columnratio{0.31}
\setlength{\columnsep}{2.2em}
\setlength{\columnseprule}{4pt}
\begin{paracol}{2}


\begin{leftcolumn}
%---------------------------------------------------------------------------------------
%	META IMAGE
%----------------------------------------------------------------------------------------
\includegraphics[width=\linewidth]{resources/profiel.jpeg}	%trimming relative to image size

%---------------------------------------------------------------------------------------
%	META SKILLS
%----------------------------------------------------------------------------------------
	\fcolorbox{white}{white}{\begin{minipage}[c][2.5cm][c]{1\mpwidth}
		\LARGE{\textbf{\textcolor{maincol}{Apurv Saha}}} \\[4pt]
		\normalsize{ \textcolor{maincol} {Robotics | Computer Vision | Machine Learning } }
\end{minipage}} \\

\cvsection{Skills}

\cvskill{Software development} {2+ yrs} {1} \\[-2pt]

\cvskill{ROS/ROS2} {2+ yrs.} {1} \\[-2pt]

\cvskill{Python} {2+ yrs.} {1} \\[-2pt]

\cvskill{C++} {1+ yrs.} {0.5} \\[-2pt]

\cvskill{Devops} {1+ yrs.} {0.5} \\[-2pt] 

\cvskill{OpenCV} {1+ yrs.} {0.5} \\[-2pt]

\cvskill{Machine Learning} {1+ yrs.} {0.5} \\[-2pt]

\cvskill{Gazebo} {1+ yrs.} {0.5} \\[-2pt]

\cvskill{React JS} {0.6+ yrs.} {0.25} \\[-2pt] 

\cvskill{Flutter} {0.6+ yrs.} {0.25} \\[-2pt] \\

\cvskill{English} {C1} {0.9} \\[-2pt]

\newpage
%---------------------------------------------------------------------------------------
%	EDUCATION
%----------------------------------------------------------------------------------------
\cvsection{Education}

\cvmetaevent
{05/2015 - 05/2019}
{Bachelor of Technology}
{International Institute of Information Technology }
{\textit{• ECE   • Robotics} \newline Bachelor's thesis: Research of Deep Learning architecture for MPC in Carla.}

\cvmetaevent
{05/2013 - 05/2015}
{95.0/100} 
{Shri Chaitanya higher secondry school}
{\textit{Maths • Physics • Chemistry • English • Computer Science}.}


%
%\cvsection{Projekte}

%	\cvlist{
%		\item \hyperlink{https://github.com/philipempl/ether-twin}{\textbf{Ether-Twin.}}\\ Ethereum Applikation für Digital Twins.
%		\item \hyperlink{https://github.com/philipempl/Peter-Pan}{\textbf{Peter Pan.}}\\ Koch-App (t.b.a.).
%		\item \hyperlink{https://github.com/philipempl/Innovation-Tool}{\textbf{Innovation Tool.}}\\ Webcrawler für \hyperlink{https://ibi.de/}{Ibi}.
%		\item \hyperlink{https://github.com/philipempl/cozone}{\textbf{COZONE.}} \\ Soziales Netzwerk (t.b.a.).
%		\item \hyperlink{https://github.com/geritwagner/enlit}{\textbf{ENLIT.}}\\ Exploring new Literature (Bachelorarbeit).
%		\item \textbf{Crowdfunding.} \\Modul mit \hyperlink{https://senacor.com/}{Senacor} für \hyperlink{https://www.paydirekt.de/}{paydirekt}.
%		}

\cvsection{Interests}

\icontext{CaretRight}{12}{Guitarist}{black}\\[6pt]
\icontext{CaretRight}{12}{Drones}{black}\\[6pt]
\icontext{CaretRight}{12}{Fitness}{black}\\[6pt]
\icontext{CaretRight}{12}{Open Mic}{black}\\[6pt]
\icontext{CaretRight}{12}{Travel}{black}\\[6pt]




\cvsection{Contact}

\icontext{MobilePhone}{16}{+91 8517087325}{black}\\[6pt]
\iconemail{Envelope}{16}{apurv.robotics@gmail.com}{XXXX@XXXXX.XX}{black}\\[6pt]
\iconhref{Github}{16}{github.com/Apurv354}{https://github.com/Apurv354}{black}\\[6pt]
\iconhref{Twitter}{16}{apurv.saha}{https://twitter.com/apurv_saha}{black}\\[6pt]

	
%\cvqrcode{0.3}

\end{leftcolumn}
\begin{rightcolumn}
%---------------------------------------------------------------------------------------
%	TITLE  HEADER
%----------------------------------------------------------------------------------------


%---------------------------------------------------------------------------------------
%	PROFILE
%----------------------------------------------------------------------------------------
\cvsection{Biography}
\vspace{4pt}

\cvtextI{Passionate Robotics engineer skilled in Computer Vision and Machine Learning with 2+ years of experience on developing software for robot's and deploying them using cloud services. My commit message are as clear as my code and I have always tried to  re base my learning on top of latest technology available. 

Apart from my software experiences I have hand's on experience on making ground vehicles and quad copters for various application and as a hobby. I am also an active open source contributor and have recently started an organization called Drone-Cloud-Automation which intends to build basic software for drone enthusiasts who wish to make a product out of it. 

}


%---------------------------------------------------------------------------------------
%	WORK EXPERIENCE
%----------------------------------------------------------------------------------------

\vspace{10pt}
\cvsection{Work experience}
\vspace{4pt}


\cvevent
{02/2020 - today}
	{Software Engineer Robotics  (Contract)}
	{Rapyuta Robotics\newline Tokyo,Japan}
	{Working on building ROS/ROS2 based software stack and deploying them to world's first cloud robotics platform rapyuta.io.
	\begin{itemize}[leftmargin=*]
    \item Working on building ROS/ROS2 module to lift pallets using Edge AI. Major responsibilities include developing the monitoring package's and masking out the unneeded information creating CI pipeline and creating multi architecture based docker images.  
    \item Worked on building and testing ROS based software components for navigation and planning in forklifts for warehouse automation and simulating them in Gazebo. Implemented a set of automated test cases for forklifts in warehouses on rapyuta.io platform. 
    \end{itemize}}

	
	

\cvevent
{05/2019 - 02/2020}
	{Robotics Software Developer}
	{Third Space Auto\newline Espoo, Finland}
	{Worked on building ROS based software stack for UAV's also involved in designing and testing algorithm's on UAV made of Rapberry PI and PX4 flight controller. 
	\begin{itemize}[leftmargin=*]
    \item Responsible for designing algorithm for precision landing on UAV's using ROS based packages. Creating Gazebo environment for testing. Worked on various control algorithm to increase stability while landing.   
    \item Research on Swarm drones and tried to simulate and study few of the existing methodologies also have hands on experience on building drones using PX4 and Ardupilot and integrating various sensors on them. 
    \end{itemize}}


   	
   	
   	

	
\cvevent
{12/2019 - 05/2020}
	{Robotics Research Internship}
	{Continental Automotive\newline Bangalore, India}
	{Worked on researching and implementing Control Algorithms on CARLA simulator. Contributed to develop a general framework for implementing Machine Learning on Autonomous Vehicle Application . 
	\begin{itemize}[leftmargin=*]
    \item The idea was to find an approach of implementing Model Predictive Controller fused with Machine Learning where the optimisation could be more enhanced and accurate to what we see in classical approaches.   
    \item Contributed to writing script for python based generic framework on top of tensorflow API which will have the ability to implement basic features of L3 autonomy in self-driving cars. 
    \end{itemize}}




\cvevent
{04/2019 - 06/2019}
	{Robotics Internship}
	{The Inkers \newline Bangalore, India}
	{Worked on implementing control algorithm's for home cleaning robot and did research on Deep reinforcement learning on self-balancing bot.  
	\begin{itemize}[leftmargin=*]
    \item Worked on beagle-bone to research and implement various localisation and mapping techniques followed by control algorithms. The idea was to make a robot which can work in 3 modes manual, semi-autonomous and autonomous.      
    \item Got an opportunity to read research papers on Deep reinforcement learning and worked on implementing Q learning on self-balancing robot using V-Rep simulator and contributed actively to some of the open source repositories working on ML. 
    \end{itemize}}




\cvsection{Achievements}

\begin{itemize}[leftmargin=*]
\item Runner up - Hack2020 conducted by Adlink \& Intel for the idea of using edge AI to solve problems in warehouse automation. The idea implementation was performed on Adlink Vizi Ai kit + OAK D cameras by Opencv.
\item Semi-Finalist in DRUSE (DRDO Robotics and unmanned system exposition) - The idea was make to an application for Intelligent surveillance system using Drones which uses object detection algorithms on suspected/potential threats .
\item All India Finalist in Techgium 2018 - The idea was to track anonymous/suspicious static objects in crowded place using Drones. Made custom drone with PX4 + Raspberry Pi with Raspicam and used control algorithms with Image processing and deep learning.
\end{itemize}


\cvsection{Publications}

\begin{itemize}[leftmargin=*]
\item Author, Apurv Saha. \& Author, Akash Kumar. \&  Author Aishwary Kumar Sahu.  "FPV drone with GPS used for surveillance in remote areas". In: \textit{2017 Third International Conference on Research in Computational Intelligence and Communication Networks (ICRCICN)}, Kolkata, Nov-17, 2017.
\item Author, Apurv Saha. \& Author, Akash Kumar. \&  Author Aishwary Kumar Sahu. "Face Recognition Drone". In: \textit{Proceedings of the 2018 3rd International Conference for Convergence in Technology (I2CT)}, Pune, June 12, 2019.
\end{itemize}
% hofixes to create fake-space to ensure the whole height is used
\mbox{}
\vfill
\mbox{}
\vfill
\mbox{}
\vfill
\mbox{}
\vfill
\mbox{}
\vfill
\mbox{}
\vfill
\mbox{}


Bengaluru, \today     \hspace{1cm}   \hrulefill

\hspace*{30mm}\phantom{Bengaluru, \today }Apurv Saha

\end{rightcolumn}
\end{paracol}


\end{document}